% !TEX TS-program = xelatex
% !TEX encoding = UTF-8 Unicode
% !Mode:: "TeX:UTF-8"

\documentclass{resume}
\usepackage{zh_CN-Adobefonts_external} % Simplified Chinese Support using external fonts (./fonts/zh_CN-Adobe/)
% \usepackage{NotoSansSC_external}
% \usepackage{NotoSerifCJKsc_external}
% \usepackage{zh_CN-Adobefonts_internal} % Simplified Chinese Support using system fonts
\usepackage{linespacing_fix} % disable extra space before next section
\usepackage{cite}

\begin{document}
	\pagenumbering{gobble} % suppress displaying page number
	
	\name{李睿}
	
	\basicInfo{
		\email{leereborn92929@gmail.com} \textperiodcentered\ 
		\phone{(+1) 613-277-1820} \textperiodcentered\ 
		\linkedin[leereborn]{https://www.linkedin.com/in/rui-li-a193a799/}}
	
	\section{\faGraduationCap\  教育背景}
	\datedsubsection{\textbf{卡尔顿大学}, 渥太华, 加拿大}{2014 -- 2018}
	\textit{荣誉学士}\ 计算机科学
	
	\section{\faUsers\ 工作经历}
	\datedsubsection{\textbf{Espial Group Inc.} 渥太华, 加拿大}{2017年1月 -- 2017年8月}
	\role{嵌入式开发实习}{}
	\begin{itemize}
		\item 协助移植机顶盒客户端代码到RDK平台
		\item 自动化每晚构建部署过程
		\item 为RDK开源社区贡献代码
	\end{itemize}
	
	\datedsubsection{\textbf{Corsa Technology}, 渥太华, 加拿大}{2016年9月 -- 2016年12月}
	\role{软件开发实习}{}
	\begin{itemize}
		\item 开发完整化测试python脚本代码
		\item 搭建并基准测试 Open vSwitch 
	\end{itemize}
	
	\datedsubsection{\textbf{卡尔顿大学}, 渥太华, 加拿大}{2016年7月 - 2017年4月}
	\role{助教}{}
	%\begin{onehalfspacing}
	\begin{itemize}
		\item 协助教授批改作业并解答学生疑问
		\item 助教课程包括:计算机科学概论,数据机构与算法,离散数学
	\end{itemize}
	%\end{onehalfspacing}
	
	% Reference Test
	%\datedsubsection{\textbf{Paper Title\cite{zaharia2012resilient}}}{May. 2015}
	%An xxx optimized for xxx\cite{verma2015large}
	%\begin{itemize}
	%  \item main contribution
	%\end{itemize}
	
	\section{\faCogs\ IT 技能}
	% increase linespacing [parsep=0.5ex]
	\begin{itemize}[parsep=0.5ex]
		\item 编程语言: Java == Python > Javascript > Bash > C++
		\item 软件库:Keras,Numpy,OpenCV,Spark,Lucene,Crawler4j,Tika,NodeJS
		\item 软件: Linux(Ubuntu), \LaTeX,Eclipse,Android Studio,Git,Jenkins
		\item 开发: 基于神经网络的图像识别,本地文件检索,Web开发,Android
	\end{itemize}
	
	\section{\faCogs\ 项目经历}
	\datedline{面部性别与年龄识别}{2018年3月}
	\begin{itemize}
		\item 设计、训练并实现用于面部性别与年龄识别的卷积神经网络
		\item 开发了通过摄像头实时检测性别与年龄的软件
	\end{itemize}
	\datedline{本地搜索引擎}{2018年2月}
	\begin{itemize}
		\item 编写网页爬虫抓取网站内容并存储到本地数据库,应用Lucene建立本地索引
		\item 实现Page Rank算法
		\item 通过RESTful接口为用户提供服务
	\end{itemize}
	\datedline{安桌聊天软件}{2017年10月}
	\begin{itemize}
		\item 开发基于Google Firebase的群聊软件,支持发送文字及图片
	\end{itemize}
	\datedline{Unix Log分析Web应用}{2016年3月}
	\begin{itemize}
		\item 实现基于Node.js的服务器支持上传Unix系统日志文件并存储于数据库
		\item 设计并实现前端网页应用支持上传,查询,显示系统日志
	\end{itemize}
	
	\section{\faInfo\ 其他}
	% increase linespacing [parsep=0.5ex]
	\begin{itemize}[parsep=0.5ex]
		\item GitHub: https://github.com/leereborn
		\item 语言: 英语 - 熟练
	\end{itemize}
	
	%% Reference
	%\newpage
	%\bibliographystyle{IEEETran}
	%\bibliography{mycite}
\end{document}
